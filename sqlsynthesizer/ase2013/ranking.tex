\subsection{Candidate Ranking}
\label{sec:ranking}


It is possible that multiple SQL queries satisfying
the given input-output examples will be returned.
This may adversely impact end-users who want to
perform simple query tasks but now need
to select the query of their intent.
To alleviate this problem, we employ
the Occam's razor principle, which states that the
simplest explanation is usually the correct one, to
rank a more likely query higher in the output list.
A simpler query is less likely to overfit the given examples
than a complex query, even when both of them
can transform the example input to the example output.


%We define a comparison scheme between different
%SQL queries by defining a partial order between them. Some of
%these choices are subjective, but have been observed to work well.
A SQL query is simpler than another one if it uses
fewer query conditions (including conditions in the \CodeIn{Having}
and \CodeIn{from} clauses) or the expressions (including
aggregates) in each query condition are pairwise simpler
(e.g., expression \CodeIn{Count(student\_id)} is simpler than
\CodeIn{Count(Distinct student\_id)}.
Simpler query conditions suggests the extraction logics
are more common and general.

In our implementation, \ourtool computes a cost for each
query, and prefers queries with lower costs. The cost
for a SQL query is computing approximately by summarizing
the number of conditions, aggregates,
and other expressions
appearing in the \CodeIn{group by} and \CodeIn{order by} clauses.
This heuristic, though fairly simple, has been observed
to work well.
Figure~\ref{fig:rank} shows an example.


\begin{figure}[t]
\centering
 \includegraphics[scale=0.80]{rankexample}
 \vspace{-3mm}
\Caption{{\label{fig:rank} Illustration of \ourtool's
query candidate ranking heuristic. \ourtool produces two
queries for the given input-output examples. Based on
the heuristic in Section~\ref{sec:ranking}, the first query
differs from the second query by using simpler conditions,
and thus is ranks higher.
}}
\end{figure}

