
\section{Related Work}
\label{sec:related}

This section discusses two categories
of closely-related work on reverse query processing 
and automated program synthesis.


\subsection{Reverse Query Processing}

Reverse query processing is a well-known problem in the database
community~\cite{Zloof:1975, Tran:2009, DasSarma:2010}. Zloof's work on \textit{Query by Example} (QBE)~\cite{Zloof:1975}
provided a high-level query language and an intuitive
form-based Graphical User Interface for
writing database queries. To use the QBE system, 
users must formulate their queries with the
provided language and fill in the appropriate
skeleton tables on its interface.
\todo{exactly what we need to avoid.}



Tran et al.~\cite{Tran:2009} proposed a technique, called \textit{Query By Output} (QBO), 
to infer SQL queries from input-output examples. \todo{say}
However, their techniques can only infer simple select-project-join queries,
and such queries can only be applied to satisfy the example in \todo{some example},
and such queries cannot be applied to many of the other real-world cases that we
studied (Section~\ref{}). The key limitation of the QBO technique is
that it only considers table values in the original input tables as
features \todo{need to read}


Recently, Sarma et al.~\cite{DasSarma:2010} studied the \textit{View Definitions Problem} (VDP).
With a rather different goal than \ourtool,
VDP aims to to find the most
succinct and accurate view definition, when
the view query is restricted to a specific family of queries.
VDP can be solved as a special case in \ourtool where there is only 1
input table with no joins nor projections.
In addition, the main contribution of Sarma et al's work
is the complexity analysis of three variants of the view
definitions problem, and there are no tool implementationand empirical study
for their proposed technique.

%Compared to \ourtool, there are three key differences:
%(1) VDP aims to infer a relation from a large representative
%example view, while \ourtool attempts to infer a SQL query from a
%set of few input-output examples \todo{revise below}
%(which is a critical usability aspect for end-users who lack adequate
%SQL experience). (2) the technique proposed by Sarma et al.~\cite{DasSarma:2010} does
%not consider join or projection
%operations and assumes the output view to be a strict subset of the input
%database with the same schema\todo{revise}, while our work eliminates these
%assumptions, and considers both join and projection operations.\todo{revise above}



\todo{what is the limitation of non program sythesis, why
sythensis would be useful.}

\subsection{Automated Program Synthesis }


Program synthesis~\cite{Gulwani:2010:DPS} aims to create an executable program
in some underlying language from specifications that can range
from logical declarative specifications to examples or
demonstrations~\cite{Harris:2011, singh:2012, Gulwani:2011,
Kandel:2011, Fisher08Pads,Lau:2003:PDU, Lau:2000:VSA, Barbosa:2010:MLA, Arasu:2009:LST}.
It has been used recently for many applications
such as synthesis of efficient low-level code~\cite{Solar-Lezama:2005},
data structure manipulations~\cite{Fisher:2008},
geometry constructions~\cite{Gulwani:2011:SGC},
snippets of excel macros~\cite{Harris:2011},
relational data representations~\cite{Barbosa:2010:MLA, Arasu:2009:LST} and string
expressions~\cite{singh:2012, Gulwani:2011}.


The PADS~\cite{Fisher:2008} system takes a large sample
of unstructured data and infers a
format that describes the data. Users then manually
define tools for data in the format.
Similarly, the Wrangler tool, developed in the HCI community,
provides a nice visual programming-by-demonstration
interface to table transformations for data cleaning~\cite{Kandel:2011}.
Theses techniques, though well-suited for their intended tasks,
are insufficient for the database query tasks.
For example, while text extraction and mass editing
are quite valuable for data transformation, the above tools
lack other needed operations such as table joinning,
aggregation, and query condition inference.
\todo{explain}


Harris and Gulwani described a system for learning excel
spreadsheet transformations macros from an example
input-output pair~\cite{Harris:2011}. Given input and output
tables, their system can infer an excel macro that filters,
duplicates, and/or reorganizes table cells to generate the output table.
\ourtool differs in multiple respects.
First, excel macros have significantly different
abstractions and semantics than the SQL language.
An excel macro can express a variety of table re-shaping operations
(e.g., \todo{an example}), but
are not suitable to formulate database queries.
Second, Harris and Gulwani's approach treats table cells
as atomic units, and thus has limited expressiveness.
It can only transforms one table to another, but
does not support queries involving table joining, aggregation,
or column projection. \todo{takk avbout the example}


%By contrast, \ourtool provides
%an interface based on examples, which is more
%friendly to end users. \todo{However, }


Some recent work proposed query recommendations systems to reduce
the obstacles to using relational databases~\cite{Howe:2011, Khoussainova:2010}. 
SQLShare~\cite{Howe:2011} is a cloud-based service that allows
users to upload their data and SQL queries. 
Each saved query is also registered as a view, allowing other users
to compose and reuse. SnipSuggest~\cite{Khoussainova:2010} is a SQL autocompletion
system. As a user typs a query, SnipSuggest mines existing query
logs to recommend relevant clauses or SQL snippets (e.g., the table
names for the \CodeIn{from} clause) based on the partial query that
the user has typed so far.
Compard to \ourtool, both SQLShare and SnipSuggest hypothesize that existing query
logs (by the user herself or other users) contains valuable information,
as a user articulates an increasingly larger fragment of a query.
However, such query logs are often not available to most users.
\ourtool eliminates this assumption by directly asking users to
draw the input-output examples.
\todo{the difficult of start writing queries.}


Cheung et al~\cite{abs-1208-2013} presented a technique to infer SQL
queries from imperative code. Their technique automatically identifies
fragments of application logic (written in an imperative language
like Java) that can be pushed into SQL queries. 
Compared to \ourtool, their work is designed for developers
to improve a database application's performance by re-writing fragments of
logic code into SQL queries, rather than helping novice end-users
write correct SQL queries from scratch. In addition,
\todo{focus on performance...}

%also cite this~\cite{Cheung:2012:UPS}, combining machine learning with program synthesis.
