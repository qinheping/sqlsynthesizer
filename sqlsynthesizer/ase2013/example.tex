\section{Motivating Example}
\label{sec:example}

Consider the following SQL question picked up from a classic
database textbook~\cite{cowbook}: \textit{given a \CodeIn{student} table (Figure~\ref{tbl:student})
and an \CodeIn{enrolled} table (Figure~\ref{tbl:enrolled}), find out the name and max score of the
students whose level is senior and enrolled in more than 3 courses}.


\begin{figure*}[t]
  \centering
  \includegraphics[scale=0.70]{motivating}
  \vspace*{-1.0ex}\caption {{\label{fig:motivating} ss.
}}
\end{figure*}


The question's description is quite simple.
For a novice user, although they have a clear
intention of what the query should do, the answer (Figure~\ref{fig:expected_sql}) may
not be that straightforward. 

Despite the possible difficult in writing a correct SQL query,
a user could still easily draw
two input tables (Figure~\ref{tbl:student} and Figure~\ref{tbl:enrolled})
and one output table (Figure~\ref{tbl:output}) that fulfill the
SQL question.

In the \CodeIn{student} table, column {\CodeIn{Student\_key}} with
\CodeIn{String} type serves as the primary key. 
%Columns {\CodeIn{Student\_name}} and {\CodeIn{Level}} are \textsf{String}-type.
In the \CodeIn{enrolled} table, both columns {\CodeIn{Student\_key}} and
{\CodeIn{Course\_key}} are two foreign keys, and column {\CodeIn{Score}}
with \CodeIn{Integer} type keeps students' scores on their enrolled courses.

In the output table, the first column  {\CodeIn{Student\_name}}
comes from table \textit{student}, and the second column {\CodeIn{Max\_score}}
is a aggregation attribute.% summarizing the with type of \CodeIn{Integer}.

Having the input and output examples, our technique successfully synthesizes
the desirable SQL query as shown in Figure~\ref{fig:expected_sql}.
