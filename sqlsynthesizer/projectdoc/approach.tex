
\section{Approach}
\label{sec:approach}

Figure~\ref{fig:syntax} defines the syntax of the supported
SQL language subset.

\newcommand{\q}{\langle query\rangle}
\newcommand{\db}{\langle db\rangle}
\newcommand{\pat}{\langle pat\rangle}
\newcommand{\bug}{\langle bug\rangle}
\newcommand{\dist}{\langle distance\rangle}
\newcommand{\sem}[1]{\llbracket #1\rrbracket}
\newcommand{\lit}[1]{\texttt{#1}}

\newcommand{\column}{\langle column\rangle}
\newcommand{\dbtable}{\langle table\rangle}
\newcommand{\cond}{\langle cond\rangle}
\newcommand{\op}{\langle op\rangle}
\newcommand{\e}{\langle expr\rangle}

\begin{figure}[t]
%\scriptsize{%
\footnotesize%
\begin{align*}
\q ::= {} 
	& \texttt{ SELECT } \column^+ \texttt{ FROM } \dbtable^+ \\
        & \texttt{ WHERE } \cond^+ \\ 
	&  \texttt{ GROUP BY } \column^+ \texttt{ HAVING } \cond^+\\
\dbtable::= {} &\ atom \\
\column ::= {} &\ \dbtable.atom\\
\cond ::= {} &\ \ \cond \;\texttt{\&\&}\; \cond \\ 
    & |\ \cond \;\texttt{||}\; \cond \\
    & |\ \texttt{(}\;\cond\;\texttt{)} \\
    & |\ \e \;\op\; \e \\
\op ::= {} &\ \ \texttt{=} \;\;|\;\; \texttt{>}  \;\;|\;\; \texttt{<}\\
\e ::= {} &\ \ const \;\;|\;\; \column  \;\; \\
    & |\ sum(\column) \;\;|\ max(\column) \;\;|\ min(\column) 
\end{align*}
\normalsize%
\caption{Syntax of the supported SQL language subset.}
\label{fig:syntax}
\end{figure}
