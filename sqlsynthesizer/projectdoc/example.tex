%\section{A Motivating Example}
%\label{sec:example}

Consider the following SQL question picked up from a classic
database textbook~\cite{cowbook}: \textit{given a \CodeIn{student} table (Figure~\ref{tbl:student})
and an \CodeIn{enrolled} table (Figure~\ref{tbl:enrolled}), find out the name and max score of the
students whose level is senior and enrolled in more than 3 courses}.



\begin{figure}[t]
	\centering
\begin{tabular}{|c|c|c|}
\hline
 Student\_key& Student\_name & Level\\
\hline
 0001 & Adam & senior \\
 \hline
 0002 & Bob & junior \\
 \hline
 0003 & Erin & senior \\
 \hline
 0004 & Rob & junior\\
 \hline
 0005 & Dan & senior \\
 \hline
 0006 & Peter & senior \\
 \hline
 0007 & Sai & senior \\
 \hline
\end{tabular}
	\caption{The example input table: student
for the motivating example described in
Section~\ref{sec:example}.}
	\label{tbl:student}
\end{figure}


\begin{figure}[t]
	\centering
\begin{tabular}{|c|c|c|}
\hline
 Student\_key& Course\_key& Score \\
\hline
 0001 & 001 & 4 \\
\hline
 0001 & 002 & 2 \\
\hline
 0002 & 001 & 3 \\
\hline
 0002 & 002 & 2 \\
\hline
 0002 & 003 & 3 \\
\hline
 0003 & 002 & 1 \\
\hline
 0004 & 001 & 4 \\
\hline
 0004 & 003 & 4 \\
\hline
 0005 & 002 & 5 \\
\hline
 0005 & 003 & 2 \\
\hline
 0005 & 004 & 1 \\
\hline
 0006 & 002 & 4 \\
\hline
 0006 & 004 & 5 \\
\hline
 0007 & 001 & 2 \\
\hline
 0007 & 003 & 3 \\
\hline
 0007 & 004 & 4 \\
 \hline
\end{tabular}
	\caption{The example input table: enrolled
for the motivating example described in
Section~\ref{sec:example}.}
	\label{tbl:enrolled}
\end{figure}	

\begin{figure}[t]
	\centering
\begin{tabular}{|c|c|}
\hline
 Student\_name & Max\_score \\
\hline
 Dan & 5 \\
\hline
 Sai & 5 \\
 \hline
\end{tabular}
	\caption{An example output for the motivating example described in
Section~\ref{sec:example}.}
	\label{tbl:output}
\vspace{3mm}
\end{figure}		


\begin{figure}[t]
\begin{CodeOut}
\begin{alltt}
\textbf{select student.Student\_name, max(enrolled.Score)
from student, enrolled
where student.Student\_key = course.Student\_key
      and student.Level = `senior'
group by student.Student\_key
having count(enrolled.course) > 3}
\end{alltt}
\end{CodeOut}
\vspace{-5mm}
	\caption{The expected for the motivating example described in
Section~\ref{sec:example}.}
	\label{fig:expected_sql}
\end{figure}

The problem's description is simple.
For a novice user, although they have a clear
intention of what the query should do, the answer (Figure~\ref{fig:expected_sql}) may
not be that straightforward. 

Despite the possible difficult in writing the SQL query, the user still could draw
two input tables (Figure~\ref{tbl:student} and Figure~\ref{tbl:enrolled})
as well as the output table (Figure~\ref{tbl:output}).

In the \CodeIn{student} table, {\CodeIn{Student\_key}} with the \textsf{String} type
is the primary key. Columns {\CodeIn{Student\_name}} and {\CodeIn{Level}} are \textsf{String}-type.
In the \CodeIn{enrolled} table, both columns {\CodeIn{Student\_key}} and
{\CodeIn{Course\_key}} are two foreign keys, and column {\CodeIn{Score}}
is \textsf{Integer} type.

In the output table (Figure \ref{tbl:output}), the first column  {\CodeIn{Student\_name}}
come from table \textit{student}, and another column {\CodeIn{Max\_score}} is a aggregation
attribute with type of \textsf{Integer}.

Having the input and output examples, our technique successfully synthesizes
the desirable SQL query as shown in Figure~\ref{fig:expected_sql}.
