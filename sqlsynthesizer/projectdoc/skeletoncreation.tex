\subsection{Query Skeleton Creation}
\label{sec:skeleton}

Our technique first scans the provided input and output examples, guesses
what a target SQL query might look like, and infers a set of query skeletons
that capture the basic query structure.

A query skeleton is an incomplete SQL query, which
captures three basic query structures that could
be decided after scanning the examples: tables used in the SQL
queries, columns used to join input tables, and the projected
table columns. Other parts in a SQL query such as conditions
(including selection and having conditions), and aggregates
will be determinated in the next step (Section~\ref{sec:completion}).

%A query skeleton is an incomplete SQL query, containing
%SQL structures that can be yet decided and also including
%unknown structures as holes.


%\begin{itemize}

%\item
\vspace{1mm}
\noindent \textit{\textbf{Step 1: Determining the Table Set.}} If multiple tables are provided as inputs,
it is very likely that all provided tables will be used in constructing the desirable SQL query.
Based on this observation, we assume that
the minimum number of used tables is the number of all provided input tables. By default, the
table set $T$ used in SQL query contains all given tables. However, it is
possible that a single table will be used for multiple times.%, leading to a more complex query.
Our technique does not forbid this case, rather, we \textit{use} a single heuristic
to estimate the used tables: if the same column from one input table appears more than once in the
example output, we add the input table the same number of times to the used table set.

%we view it as a strong indicator that this table will be joined multiple times and add it to our table
%set $T$ using an alias.


%\item
\vspace{1mm}
\noindent\textit{\textbf{Step 2: Determining Joining Columns. }} Given two arbitrary tables, there exist many
ways to join them. Enumerating all possible joining conditions would introduce a huge number of joining
conditions and would quickly become intractable. We observe that, in practice, two tables are often joined \textit{only} via the following
three cases: (1) tables are joined on their primary keys that have the same (or compatible) data types (i.e., it does not
make any sense to join two tables on a Integer column and a String column). (2) tables are joined
using the columns with the same column name, such as joining a \textit{Student} table with a \textit{Course} table on the
\textit{student\_name} column. and (3) two columns that have the data type, and have a large portion of
overlapped data value in its input table can be used as a joining condition. In our tool implementation, it is trivial to check the first 
two cases. For the third case, our algorithm scans the whole given input table to check the ``value similarity''
between two arbitrary columns, and selects columns whose ``value similarity'' is above a fixed threshold as joining candidates.

%\item
\vspace{1mm}
\noindent \textit{\textbf{Step 3: Determining Output Columns.}} To identify output table columns on
which the joined result would be projected, our technique checks whether each output
table column name appears in one of the input tables. If so, we assume the matched column
from the input table is used as the output column. Otherwise, the output column
must be produced by using aggregation. Our technique keeps track of those aggregation columns
and search for proper aggregates in the next phases (Section~\ref{}). 

%After determining the table set and joining columns,
%the next step is to identify potential column names on which the result would be projected. If a
%column in the output table  does not appear in any input table's column list, this output column must
%be produced by aggregation. Our algorithm keeps track of these columns and appends a \CodeIn{group by} ... \CodeIn{having} ...
%clause to the query skeleton.

%\end{itemize}
\vspace{1mm}

In summary, this step infers two parts as a query skeleton: a table set used in producing the SQL query, a set of joining conditions
used to connect tables from the table set, and a list of columns used to project the output data.
It is worth noting that the results obtained from the above steps are not \textit{safe} in
terms that they may miss some valid SQL queries. 

%We made the above assumption for the sake of tractability,
%since in theory, the bound of table number in a SQL query is $O(n_t!)$, where $n_t$ is the number of given tables;
%while the bound of possible number of join is $O(c_t^2)$ and the bound of the number of conditions is $O(n_t!n_tc_t^2)<O(n_t^3c_t^2)$.

%\subsubsection{Inferring Output Table Schema}

%Lacks schema

\subsubsection{Example}

an example

