\newcommand{\q}{\langle query\rangle}
\newcommand{\db}{\langle db\rangle}
\newcommand{\pat}{\langle pat\rangle}
\newcommand{\bug}{\langle bug\rangle}
\newcommand{\dist}{\langle distance\rangle}
\newcommand{\sem}[1]{\llbracket #1\rrbracket}
\newcommand{\lit}[1]{\texttt{#1}}

\newcommand{\column}{\langle column\rangle}
\newcommand{\dbtable}{\langle table\rangle}
\newcommand{\cond}{\langle cond\rangle}
\newcommand{\op}{\langle op\rangle}
\newcommand{\e}{\langle expr\rangle}
\newcommand{\ce}{\langle cexpr\rangle}

\begin{figure}[t]
%\scriptsize{%
\footnotesize%
\begin{align*}
\q ::= {} 
	& \texttt{ SELECT } \e^+ \texttt{ FROM } \dbtable^+ \\
        & \texttt{ WHERE } \cond^+ \\ 
	&  \texttt{ GROUP BY } \column^+ \texttt{ HAVING } \cond^+\\
\dbtable::= {} &\ atom \\
\column ::= {} &\ \dbtable.atom\\
\cond ::= {} &\ \ \cond \;\texttt{\&\&}\; \cond \\ 
    & |\ \cond \;\texttt{||}\; \cond \\
    & |\ \texttt{(}\;\cond\;\texttt{)} \\
    & |\ \ce \;\op\; \ce \\
\op ::= {} &\ \ \texttt{=} \;\;|\;\; \texttt{>}  \;\;|\;\; \texttt{<}\\
\ce ::= {} &\ \ const \;\;|\;\; \column  \;\; \\
\e ::= {} & \ce \\
    & |\ sum(\column) \;\;|\ max(\column) \;\;|\ min(\column) 
\end{align*}
\normalsize%
\caption{Syntax of the supported SQL language subset.}
\label{fig:syntax}
\end{figure}


\section{Language}
\label{sec:langsubset}

Before presenting our SQL query inference algorithm, we first describe
the supported SQL language subset.

Figure~\ref{fig:syntax} defines the syntax of the supported, which
is a subset of the standard query language SQL. This language subset
supports common query operations using multiple tables (i.e.,
\CodeIn{select} ... \CodeIn{from}... \CodeIn{where}..), and standard
boolean operators as predicates. The language
subset shares the same semantics with the standard SQL language.
Different than existing work~\cite{DasSarma:2010}, this language supports
joining operations across multiple tables.  In addition, the language includes
widely used database operations such as \CodeIn{group by},
\CodeIn{order by}, and \CodeIn{having}, as
well as a few common aggregation functions such as \CodeIn{sum},
\CodeIn{max}, and \CodeIn{min}.

Compared to the full SQL specification, this language subset is far from complete.
However, there is a tradeoff between the completeness (or expressiveness) of a language,
and the complexity of finding a simple consistent program (here, a SQL query)
within that space defined by the language. In general, the more expressive
a language, the harder the task of effectively finding an answer. When
designing the language subset, we balanced three inter-related principles:

\begin{itemize}

\item \textit{Expressiveness}. The language subset be able
to represent a reasonable set of SQL queries in practice.

\item \textit{Generality}. The language should resemble
the existing SQL language and should not impose additional
or different semantics.

\item \textit{Tool utility}. Inferring a consistent query
using this language subset should be easy to leverage by
domain knowledges and other analyses.

\end{itemize}

As these principles stand in tension, our supported language subset
represents a sweet spot that addressed each of them to a reasonable
extent. As we will show in Section~\ref{sec:evaluation}, this
language subset is expressive enough for most SQL query exercise
after a widely-used textbook and many online forum answers.

