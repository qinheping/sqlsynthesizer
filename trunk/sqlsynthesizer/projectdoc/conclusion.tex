\section{Conclusions and Future Work}
\label{sec:conclusion}

General purpose programming platforms has never been easy. Many end-users
who are non-professional programmers are still mostly stucked with the process
of \textit{how} to accomplish a certain task by receiving 
step-by-step, detailed, and syntactically correct instructions, instead of simply describing
\textit{what} the task is. \textit{Programming by examples} has the potential
to change this landscape, when targeted for the right set of people for the
right set of problems.

In this paper, we study the problem of automated SQL query sythensis
from input-output examples, present a practical technique that
infers desirable SQL queries that fulfill end-users' intents, and
developed a usable programming tool.
To show the practical usefulness of our technique, we applied our tool
to 6 SQL exercises from a classic database textbook as well as
5 non-trivial SQL problems from online forums by real end-users. 
The results are promising, showing that our technique can automate a variety of database
query tasks.

The source code of our tool implementation is available at: \\
\url{http://sqlsynthesizer.googlecode.com}

\vspace{1mm}

Besides general issues such as performance and ease of use, our future
work will concentrate on the following topics:

\textbf{Enrich the supported SQL subset.} The technique 
presented in this paper focuses on a widely-used SQL subset.
It may miss some important language features such as
\CodeIn{exists}, and thus fail to infer meaningful queries
in some scenarios when such missing features must be used.
A possible way to remedy this problem is to enrich the
supported SQL subset, and design a more efficient algorithm
to infer more potential answers.

\textbf{Illustration of the synthesis steps.} Besides
producing a final result, end-users may also be interested in knowing
how a SQL query is inferred.  Showing detailed inference steps (visually) permits
end-users to better understand the whole process, spot possible errors earlier,
and also makes the tool more usable. We
plan to investigate how to apply recent advance in data visulaization
~\cite{Kandel:2011}
to the context of program synthesis.

\textbf{Noise detection and tolerance in users' inputs.} The current technique
requires users to provide a noise-free input-output example. The
inference algorithm will declare failure when it fails to learn a valid
SQL query even in the presence of a small amount of user-input noises.
To overcome this limitation, we plan to design a more robust inference
algorithm that can attempt to identify and tolerate user-input noise, such
as inadvertent error in one input example, and even suggest a fix to
the noisy example.



\textbf{Generalization.}
We have applied our synthesis technique to one specific, but
important problem.  There is a rich history of end-user programming
problems being cast as program synthesis(e.g.,~\cite{Gulwani:2010:DPS}).  In future
work, we plan to formulate other tasks, such as repetitive file operations, and to apply our technique 
to them.

